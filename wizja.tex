\documentclass[11pt]{article}
\usepackage{polski}
\usepackage[utf8]{inputenc}
\usepackage[margin=0.78in]{geometry}
\usepackage{titling}
\usepackage{hyperref}

\setlength{\droptitle}{-4em}
\author{\vspace{-5ex}}
\date{\vspace{-11ex}}
\title{\textbf{Aplikacja mobilna \textit{"Podziel się"}}}

\begin{document}
\maketitle

\subsection*{Opis projektu}
\textit{"Podziel się"} to aplikacja mobilna przeznaczona do wzajemnego dzielenia się żywnością dla użytkowników systemów operacyjnych Android oraz iOS. Celem projektu było stworzenie platformy, która umożliwi łatwe i bezpłatne dzielenie się jedzeniem z innymi mieszkańcami w okolicy: osoby posiadające nadwyżkę artykułów żywnościowych lub gotowych potraw mogą przekazać je użytkownikom, którzy zadeklarują chęć ich odbioru.
\vspace{2mm}\\Warto podkreślić, że dotychczas powstało już kilka aplikacji typu "food sharing", które z~powodzeniem funkcjonują na terenie Wielkiej Brytanii (\textit{Olio}), Danii (\textit{Too Good To Go}) czy Stanów Zjednoczonych (\textit{Unsung}). \textit{"Podziel się"} ma na celu promować ideę dzielenia się żywnością także wśród polskich użytkowników.

\subsection*{Problem i jego rozwiązanie}
Obecnie w Polsce istnieją liczne grupy entuzjastów wymiany żywności funkcjonujące na portalach społecznościowych (przeważnie na Facebooku), jednak duża liczebność tych grup uniemożliwia skuteczną organizację wymiany jedzenia w satysfakcjonujący sposób. Aplikacja \textit{"Podziel się"} będzie wspierać dalsze funkcjonowanie tej inicjatywy oraz zapewni zwolennikom idei food sharingu platformę do efektywnej wymiany żywności, także w znacznie większej skali.

\subsection*{Kontekst społeczny}
Według współczesnych szacunków, około 1/3 wytworzonej na świecie żywności nigdy nie zostanie zjedzona. Skala problemu jest szczególnie widoczna w danych Eurostatu, z których wynika, że w Polsce marnuje się około 9 milionów ton żywności rocznie. Jej wartość sięga około 50 złotych miesięcznie w przeliczeniu na~jednego mieszkańca Polski. Badanie Millward Brown wykazało, że do wyrzucania jedzenia przyznaje się 35\% Polaków, podczas gdy wyrzucane produkty mogłyby przyczynić się do zaspokojenia podstawowych potrzeb żywieniowych niemal 3 mln osób żyjących w skrajnym ubóstwie. Nasz kraj jest piątym wśród państw Unii Europejskiej, w których marnuje się najwięcej żywności.
\vspace{2mm}\\
W krajach rozwiniętych około 50\% wyrzucanego jedzenia pochodzi z gospodarstw domowych, w których na skutek nierozważnego planowania zakupów jedzenie ulega zepsuciu i~trafia do śmietnika. Wierzymy, że~możliwe jest przeciwdziałanie temu zjawisku na poziomie lokalnym.\vspace{2mm}\\
Istnieją ruchy społeczne (dumpster diving, freeganizm, zero waste) promujące możliwie racjonalny obrót żywnością oraz uwrażliwiające społeczeństwo na problem masowego wyrzucania jedzenia.

\subsection*{Zamawiający}
Aplikacja \textit{"Podziel się"} powstała na zamówienie pani Marii Skołożyńskiej, działaczki społecznej oraz pomysłodawczyni akcji \textit{Podziel się Posiłkiem z Bezdomnymi}, podczas której mieszkańcy całej Polski mogą przekazać pozostałe po świętach (Bożego Narodzenia, Wielkanocy) posiłki do wolontariuszy, którzy zajmują się transportem jedzenia do osób potrzebujących. Pani Maria jest wolontariuszką działającą na rzecz przeciwdziałania marnowaniu jedzenia oraz główną koordynatorką akcji dystrybucji żywności. Działalność pani Marii Skołożyńskiej ma zasięg ogólnopolski; akcje gromadzenia żywności dla bezdomnych i ubogich działają obecnie w kilkudziesięciu miastach, m. in. w Warszawie, Krakowie, Wrocławiu, Trójmieście, Szczecinie, Toruniu, Bydgoszczy, Katowicach, Łodzi. W 2018 r. akcja \textit{Podziel się Posiłkiem z Bezdomnymi}, w~uznaniu dla skuteczności działania oraz promocji pozytywnych postaw społecznych, otrzymała Europejską Nagrodę Reduce Food Waste.

\subsection*{Osoby zaangażowane w projekt}
Aplikacja \textit{"Podziel się"} została opracowana oraz wdrożona przez zespół studentów informatyki Uniwersytetu Warszawskiego: Kamila Ćwintala, Szymona Gajdę, Pawła Giżkę oraz Tomasza Kanasa. Opiekę nad projektem sprawował dr Janusz Jabłonowski. Wymagania funkcjonalne zostały dostarczone zespołowi przez pomysłodawczynie projektu: p. Marię Skołożyńską oraz p. Martynę Dakowską. Podczas tworzenia aplikacji zespół programistów był wspierany przez p. Katarzynę Kubalską oraz p. Barbarę Jóźwiak, pełniące funkcję UX \& UI Designerów.

\subsection*{Grupa docelowa}
Aplikacja \textit{"Podziel się"} jest przeznaczona dla użytkowników smartfonów, którym nie jest obojętny problem marnowania żywności. \textit{"Podziel się"} może stanowić interesującą propozycję dla osób, które nie mają czasu na przygotowanie posiłku albo gotują zbyt dużo (i~jednocześnie nie chcą wyrzucać nadwyżki jedzenia). Aplikacja \textit{"Podziel się"} zrzesza społeczność osób zainteresowanych tematyką food sharingu i zero waste, które dbają o środowisko naturalne oraz racjonalną gospodarkę żywnością.

\subsection*{Opis funkcjonalności}
Aplikacja \textit{"Podziel się"} umożliwia:
\begin{itemize}
\setlength\itemsep{-0.2em}
\item bezpłatną rejestrację nowego konta,
\item uzupełnienie własnego profilu o podstawowe dane personalne,
\item dodawanie nowego ogłoszenia (opisu produktu wraz z preferowaną lokalizacją miejsca odbioru),
\item zamieszczenie w ogłoszeniu zdjęć produktu, opakowania, etykiet itp.,
\item przeglądanie dostępnych do odebrania posiłków w okolicy aktualnej lokalizacji użytkownika,
\item filtrowanie dostępnych ogłoszeń w oparciu o wprowadzone przez użytkownika kryteria i preferencje,
\item funkcję czatu między odbiorcą a darczyńcą w celu ustalenia szczegółów odbioru, zadawania pytań,
\item przechowywanie historii konwersacji z innymi użytkownikami,
\item wskazanie użytkownika (ze zbiorczej listy osób zainteresowanych ogłoszeniem), która otrzyma wystawiony~produkt/artykuł żywnościowy,
\item pośrednictwo w procesie przekazania żywności,
\item usuwanie ogłoszeń zrealizowanych pomyślnie lub ogłoszeń, dla których minęła data przydatności do~spożycia,
\item system recenzji/ocen użytkowników,
\item wyróżnianie wybranych użytkowników jako "ulubionych", dzięki czemu ich nowe ogłoszenia będą dodatkowo eksponowane na liście dostępnych ogłoszeń,
\item flagowanie ogłoszeń o nieodpowiedniej zawartości, co skutkuje powiadomieniem moderatora o~konieczności~weryfikacji zamieszczonej treści,
\item personalizację wyglądu aplikacji (np. kolor elementów interfejsu).
\end{itemize}
Aplikację wyróżnia nowoczesny interfejs graficzny, zaprojektowany specjalnie na potrzeby projektu.

\subsection*{Technologie i narzędzia}
Interfejs aplikacji powstał z wykorzystaniem frameworka React Native, umożliwiającego tworzenie aplikacji mobilnych natywnie renderowanych na systemy operacyjne Android oraz iOS. Część serwerowa \textit{"Podziel się"} oparta jest na Django, popularnym frameworku back-endowym. Aplikacja wykorzystuje relacyjną bazę danych PostgreSQL.
\vspace{2mm}\\
Podczas całego cyklu tworzenia i rozwoju aplikacji kod źródłowy oraz konfiguracja serwera (udostępniane API, model bazy danych) były przechowywane w repozytorium GitHub.

\subsection*{Bibliografia}
\begin{itemize}
\setlength\itemsep{-0.1em}
\item strona akcji społecznej "Podzielmy się", \href{www.podzielmysie.pl}{\texttt{podzielmysie.pl}}
\item materiały własne pani Marii Skołożyńskiej
\item artykuł prasowy opublikowany w Strefie Biznesu \href{https://strefabiznesu.pl/nie-wyrzucaj-jedzenia-po-swietach-podziel-sie-z-potrzebujacymi-jak/ar/13053810}{"Nie wyrzucaj jedzenia po świętach, podziel się z~potrzebującymi. Jak?"}
\item artykuł prasowy portalu Business Insider \href{https://businessinsider.com.pl/lifestyle/jedzenie/marnowanie-zywnosci-ile-ton-jedzenia-wyrzucaja-polacy/5wnn8yt}{"Polacy rocznie marnują 9 mln ton jedzenia"}
\item artykuł prasowy opublikowany w internetowym wydaniu Newsweeka \href{https://newsweek.pl/polska/foodsharing/qdnp614}{"Foodsharing – pomysł z Niemiec rozwija się nad Wisłą"}

\item statystyki dotyczące wykorzystania żywności na świecie z oficjalnej strony aplikacji  \textit{Olio}, \href{https://olioex.com}{\texttt{olioex.com}}
\item oficjalna strona internetowa projektu Reduce Food Waste, \href{www.reducefoodwaste.eu}{\texttt{reducefoodwaste.eu}}
\end{itemize}

\end{document}
