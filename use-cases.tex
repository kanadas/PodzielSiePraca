
\documentclass[12pt,a4paper,twoside]{article}
\RequirePackage[T1]{fontenc}
\RequirePackage{times}
\RequirePackage[utf8]{inputenc}
\RequirePackage[polish]{babel}


\RequirePackage{comment}


\RequirePackage{a4wide}
\RequirePackage{longtable}
\RequirePackage{multicol}
\RequirePackage{url}
\RequirePackage{enumitem}


\begin{document}


    \section{Pierwsze uruchomienie aplikacji}


    \subsection{Krótki opis}
    Gdy użytkownik pierwszy raz uruchamia aplikację, następuje ``przywitanie'', wyświetlenie podstawowych informacji o aplikacji oraz zadanie pytania o najważniejsze ustawienia.


    \subsection{Cele}
    \begin{itemize}
        \item Zachęcenie użytkownika do korzystania z aplikacji.
        \item Pomoc użytkownikowi w szybkim opanowaniu funkcji aplikacji.
        \item Uzyskanie informacji koniecznych do poprawnego działania aplikacji.
    \end{itemize}


    \subsection{Warunki wstępne}
    \begin{itemize}
        \item Aplikacja jest zainstalowana na urządzeniu użytkownika.
        \item Użytkownik jeszcze nie uruchamiał aplikacji.
        \item Użytkownik dysponuje stabilnym połączeniem internetowym.
    \end{itemize}


    \subsection{Czynności}

    \subsubsection{Czynności podstawowe}


    \begin{enumerate}
        \item Użytkownik uruchamia aplikację.
        \item Aplikacja wyświetla logo.
        \item Aplikacja wyświetla krótki film instruktażowy.
        \item Aplikacja pyta użytkownika, czy chce korzystać z geolokalizacji.
        \item Użytkownik zgadza się na korzystanie z geolokalizacji.
        \item Aplikacja wyświetla listę ogłoszeń posortowaną względem odległości od użytkownika.
    \end{enumerate}


    \subsubsection{Czynności alternatywne}


    \begin{enumerate}
        \item Użytkownik nie zgadza się na korzystanie z geolokalizacji.
        \item Aplikacja pyta użytkownika, względem którego miejsca ma sortować ogłoszenia i wyświetla interaktywną mapę.
        \item Użytkownik wybiera miejsce na mapie.
        \item Aplikacja wyświetla listę ogłoszeń posortowaną względem odległości od miejsca wybranego przez użytkownika.
    \end{enumerate}


    \subsection{Możliwe rozwinięcia}
    \subsubsection{Dodanie interaktywnego samouczka}
    Po (lub zamiast) wyświetlania filmu instruktażowego użytkownik dostanie możliwość przejścia samouczka.


    \section{Rejestracja}


    \subsection{Krótki opis}
    Utworzenie konta użytkownika w aplikacji.


    \subsection{Cele}
    Umożliwienie aplikacji identyfikacji użytkowników w celu dostarczenia im określonych funkcjonalności.


    \subsection{Warunki wstępne}
    \begin{itemize}
        \item Aplikacja jest zainstalowana na urządzeniu użytkownika.
        \item Użytkownik przeszedł przez ``Pierwsze uruchomienie aplikacji''.
        \item Użytkownik dysponuje stabilnym połączeniem internetowym.
    \end{itemize}


    \subsection{Czynności}


    \subsubsection{Czynności podstawowe}


    \begin{enumerate}
        \item Użytkownik naciska przycisk ``Załóż konto''.
        \item Aplikacja wyświetla użytkownikowi formularz rejestracyjny.
        \item Użytkownik wprowadza swoje imię, nazwisko (lub nazwę firmy), numer telefonu, adres e-mail, hasło.
        \item W polu ``Potwierdź hasło'' użytkownik wpisuje drugi raz to samo hasło.
        \item Użytkownik akceptuje regulamin.
        \item Użytkownik naciska przycisk ``Zarejestruj''.
        \item Aplikacja wysyła użytkownikowi wiadomość e-mail z linkiem aktywacyjnym.
        \item Użytkownik odbiera wiadomość e-mail i uruchamia stronę wskazaną linkiem aktywacyjnym.
        \item Aplikacja potwierdza poprawną rejestrację.
     \item Użytkownik jest domyślnie zalogowany do aplikacji na danym urządzeniu.
     \item Użytkownik może nacisnąć przycisk ``Wyloguj’’, aby się wylogować.
    \end{enumerate}


    \subsubsection{Czynności alternatywne}


    \begin{enumerate}
        \item Użytkownik:
        \begin{itemize}
            \item wprowadza drugi raz inne hasło.
            \item nie wypełnia któregoś z pól.
            \item wprowadza niepoprawny składniowo adres e-mail.
            \item nie akceptuje regulaminu.
        \end{itemize}
        \item Użytkownik naciska przycisk ``Zarejestruj''.
        \item Aplikacja powiadamia użytkownika o popełnionym błędzie i odsyła go z powrotem do edycji formularza.
    \end{enumerate}


    \subsubsection{Czynności niepoprawne}
    \begin{enumerate}
        \item Użytkownik wprowadza niepoprawny adres e-mail.
        \item Wiadomość e-mail nie przychodzi i użytkownik musi wypełnić formularz od początku.
    \end{enumerate}


    \subsection{Możliwe rozwinięcia}


    \subsubsection{Rejestracja przez Facebook}
    Dodanie przycisku umożliwiającego utworzenie konta za pomocą konta Facebook.


    \section{Logowanie}


    \subsection{Krótki opis}
    Zalogowanie się do swojego konta w aplikacji.


    \subsection{Cele}
    Umożliwienie aplikacji identyfikacji użytkowników w celu dostarczenia im określonych funkcjonalności.


    \subsection{Warunki wstępne}
    \begin{itemize}
        \item Aplikacja jest zainstalowana na urządzeniu użytkownika.
        \item Użytkownik przeszedł przez ``Pierwsze uruchomienie aplikacji''.
     \item Użytkownik posiada potwierdzone konto w aplikacji.
 \item Użytkownik nie jest zalogowany w aplikacji.
        \item Użytkownik dysponuje stabilnym połączeniem internetowym.
    \end{itemize}


    \subsection{Czynności}


    \subsubsection{Czynności podstawowe}


    \begin{enumerate}
        \item Użytkownik naciska przycisk ``Zaloguj’’.
        \item Aplikacja wyświetla użytkownikowi formularz logowania.
        \item Użytkownik wprowadza adres email i hasło, których użył przy zakładaniu konta.
        \item Użytkownik naciska przycisk ``Zaloguj’’.
        \item Aplikacja potwierdza poprawne logowanie.
     \item Użytkownik jest domyślnie zalogowany do aplikacji na danym urządzeniu.
     \item Użytkownik może nacisnąć przycisk ``Wyloguj’’, aby się wylogować.
    \end{enumerate}


    \subsubsection{Czynności alternatywne}


    \begin{enumerate}
        \item Użytkownik wprowadza błędny email lub hasło.
        \item Użytkownik naciska przycisk ``Zaloguj’’.
        \item Aplikacja powiadamia użytkownika o popełnionym błędzie i odsyła go z powrotem do formularza.
    \end{enumerate}


    \section{Filtrowanie i sortowanie listy ogłoszeń}


    \subsection{Krótki opis}
    Zmiana przez użytkownika ustawień regulujących, które wpisy pojawiają się na liście oraz w jakiej kolejności.


    \subsection{Cele}
    Zapewnienie możliwości łatwiejszego i szybszego znalezienia interesujących ogłoszeń.


    \subsection{Warunki wstępne}
    \begin{itemize}
        \item Aplikacja jest zainstalowana na urządzeniu użytkownika.
        \item Użytkownik przeszedł przez ``Pierwsze uruchomienie aplikacji''.
        \item Użytkownik znajduje się na ekranie z listą ogłoszeń.
        \item Użytkownik dysponuje stabilnym połączeniem internetowym.
    \end{itemize}


    \subsection{Czynności}


    \subsubsection{Filtrowanie}


    \begin{enumerate}
        \item Użytkownik naciska przycisk ``\#'' lub ``kategoria'' (znajdujące się nad listą ogłoszeń i podpisane ``Filtruj:'').
        \item Wyświetla się menu z listą i polem tekstowym.
        \item Użytkownik wpisuje pewien ciąg znaków.
        \item Lista aktualizuje się w czasie wpisywania tekstu.
        \item Użytkownik wybiera pole z listy.
        \item Nazwa pola (poprzedzona znakiem $\times$) zaczyna się wyświetlać pod odpowiednim przyciskiem.
        \item Lista automatycznie się aktualizuje, wyświetlając obiekty spełniające jakiś filtr każdego typu (\# i kategoria) - gdy nie ma żadnego filtru danego typu, aplikacja traktuje to tak, jakby były wszystkie możliwe.
        \item Użytkownik może nacisnąć na nazwę któregoś z aktywnych filtrów, aby go usunąć.
    \end{enumerate}


    \subsubsection{Sortowanie}


    \begin{enumerate}
        \item Użytkownik naciska przycisk ``Odległość'', ``Data przydatności'', ``Data dodania''.
        \item Lista aktualizuje kolejność wyświetlanych ogłoszeń, ale wyświetla tylko te, które nie są zbyt daleko od użytkownika. Użytkownik może zmienić maksymalną odległość wyświetlanych ogłoszeń.
    \end{enumerate}


    \subsection{Możliwe rozwinięcia}
    \subsubsection{Preferowane ogłoszenia}
    Dodanie zalogowanemu użytkownikowi możliwości wyspecyfikowania, jakie ogłoszenia go bardziej interesują (w zależności od np.\ osoby oferującej). Takim użytkownikom będzie się wyświetlał dodatkowy przycisk sortujący ogłoszenia preferując te wyspecyfikowane.


    \section{Dodawanie nowego ogłoszenia}


    \subsection{Krótki opis}
    Wprowadzenie przez użytkownika informacji potrzebnych do stworzenia nowego ogłoszenia oraz opublikowanie go na liście aktualnie dostępnych
    ogłoszeń.


    \subsection{Cele}
    Zapewnienie możliwości przeglądania nowo utworzonego ogłoszenia innym użytkownikom aplikacji.


    \subsection{Warunki wstępne}
    \begin{itemize}
        \item Aplikacja jest zainstalowana na urządzeniu użytkownika.
        \item Użytkownik posiada poprawnie założone oraz potwierdzone konto.
        \item Użytkownik jest zalogowany w aplikacji.
        \item Użytkownik dysponuje stabilnym połączeniem internetowym.
    \end{itemize}


    \subsection{Czynności}


    \subsubsection{Czynności podstawowe}


    \begin{enumerate}
        \item Użytkownik naciska oznaczony przycisk, który rozwija szufladkowe menu.
        \item Użytkownik wskazuje w menu pozycję ``Dodaj nowe ogłoszenie''.
        \item Na ekranie pojawia się formularz dodawania ogłoszenia z podpisanymi polami.
        \item Użytkownik wprowadza (w dowolnej kolejności) wymagane informacje:
        \begin{itemize}
            \item Kategoria ogłoszenia (np.\ pieczywo, mięso, przetwory, wyroby cukiernicze\ldots)
            \begin{itemize}
                \item Użytkownik rozwija przewijalną listę dostępnych kategorii.
                \item Użytkownik wskazuje kategorię, która wydaje się najbardziej odpowiednia do zawartości ogłoszenia.
            \end{itemize}
            \item Tytuł ogłoszenia, który będzie widoczny na liście ogłoszeń
            \item Właściwa treść ogłoszenia
            \item Sugerowana data przydatności do spożycia
            \begin{itemize}
                \item Użytkownik otwiera okienko kalendarza z graficznym rozkładem dni bieżącego miesiąca.
                \item W razie potrzeby użytkownik przechodzi za pomocą przycisków-strzałek do dalszych miesięcy.
                \item Użytkownik wskazuje konkretny dzień w kalendarzu.
            \end{itemize}
        \end{itemize}
        \item Użytkownik przesuwa widok ekranu na koniec formularza i naciska przycisk ``Wyślij''.
        \item Aplikacja komunikuje się z serwerem, wysyłając prośbę o dodanie wprowadzonego ogłoszenia do bazy danych.
        \item Aplikacja otrzymuje odpowiedź od serwera --- ogłoszenie zostało dodane pomyślnie.
        \item Użytkownik dostaje informację zwrotną: aplikacja wyświetla na ekranie okienko o pomyślnym dodaniu nowego ogłoszenia.
    \end{enumerate}


    \subsubsection{Czynności alternatywne}
    Użytkownik może opcjonalnie dodać zdjęcia artykułu żywnościowego, opakowania, etykiet.
    \begin{enumerate}
        \item Użytkownik wskazuje przycisk ``Dodaj nowe zdjęcie''.
        \item Aplikacja wyświetla okienko z prośbą o wskazanie źródła zdjęcia, do wyboru: załączenie zdjęcia z pamięci urządzenia (galerii)
        lub utworzenie aparatem nowego zdjęcia.
        \begin{enumerate}
            \item Jeżeli użytkownik wyrazi chęć dodania istniejącego zdjęcia, aplikacja uruchamia klasyczny interfejs przeglądania galerii
            systemu operacyjnego.
            \item Wybór opcji zrobienia nowego zdjęcia przekazuje sterowanie wbudowanej w system operacyjny aplikacji obsługującej aparat.
        \end{enumerate}
        \item Jeżeli zdjęcie ma zbyt duży rozmiar, następuje jego kompresja (po stronie aplikacji).
        \item Na ekranie formularza pojawia się miniaturka dodanego zdjęcia.
    \end{enumerate}


    \subsubsection{Czynności niepoprawne}
    We wszystkich wymienionych poniżej przypadkach na ekranie wyświetla się (jaskrawym czerwonym kolorem) stosowny komunikat. Użytkownik zostaje
    przekierowany z powrotem do ekranu edycji formularza, aby mógł wprowadzić zmiany.
    \begin{enumerate}
        \item Użytkownik nie wprowadził kompletu wymaganych informacji.
        \item Wartości niektórych pól formularza nie spełniają kryteriów akceptacji (np.\ zbyt długa treść ogłoszenia)
        \item Nastąpił błąd komunikacji z serwerem --- prośba o dodanie ogłoszenia nie została rozpatrzona.
        \item Serwer zasygnalizował błąd podczas operacji dodawania ogłoszenia do bazy danych.
    \end{enumerate}


    \subsection{Możliwe rozwinięcia}
    \subsubsection{Automatyczna klasyfikacja ogłoszeń}
    Wyświetlenie podpowiedzi co do wyboru odpowiedniej kategorii na podstawie analizy słownictwa użytego w treści ogłoszenia.


    \section{Konwersacja (czat) z innym użytkownikiem}


    \subsection{Krótki opis}
    Wykorzystanie funkcji czatu do komunikacji z innym użytkownikiem aplikacji.


    \subsection{Cele}
    \begin{itemize}
        \item Możliwość zadawania wystawcy ogłoszenia dodatkowych pytań.
        \item Udostępnienie prywatnego kanału do ustalania szczegółów odbioru produktu.
        \item Zachęcenie użytkowników aplikacji do współtworzenia aktywnej społeczności.
    \end{itemize}


    \subsection{Warunki wstępne}
    \begin{itemize}
        \item Aplikacja jest zainstalowana na urządzeniu użytkownika.
        \item Użytkownik posiada poprawnie założone oraz potwierdzone konto.
        \item Użytkownik jest zalogowany w aplikacji.
        \item Użytkownik dysponuje stabilnym połączeniem internetowym.
    \end{itemize}


    \subsection{Czynności}


    \subsubsection{Czynności podstawowe}


    \begin{enumerate}
        \item Użytkownik A otwiera ekran czatu z użytkownikiem B. Może to zrobić na kilka sposobów:
        \begin{itemize}
            \item Bezpośrednio ze strony profilowej użytkownika B.
            \item Naciskając przycisk ``Zadaj pytanie'' widocznego na ekranie ogłoszenia wystawionego przez B.
            \item Wybierając pseudonim użytkownika B z zakładki ``Moje konwersacje'' w menu szufladkowym.
        \end{itemize}
        \item W dolnej części ekranu pojawia się pole tekstowe do wpisania wiadomości oraz przycisk służący do jej wysłania. Jeżeli użytkownicy A i B rozmawiali ze sobą wcześniej, na ekranie będzie także widoczna przewijana lista poprzednich wiadomości, uporządkowana chronologicznie.
        \item Użytkownik A wpisuje treść wiadomości i wysyła ją.
        \item Nowa wiadomość zostaje dopisana do prowadzonej historii konwersacji.
        \item Użytkownik A oczekuje na odpowiedź użytkownika B:\@
        \begin{itemize}
            \item Jeżeli osoba B udzieli odpowiedzi od razu, jej treść pojawi się na ekranie użytkownika A w czasie rzeczywistym.
            \item W przypadku, gdy użytkownik A nie korzysta w danej chwili z aplikacji ``Podziel się'', otrzyma powiadomienie (notyfikację) systemu operacyjnego o nowej wiadomości od użytkownika B.
        \end{itemize}
    \end{enumerate}


    \subsubsection{Czynności niepoprawne}
    We wszystkich wymienionych poniżej przypadkach na ekranie wyświetla się okienko ze stosownym komunikatem.
    \begin{enumerate}
        \item Osoba A znajduje się na ``czarnej liście'' użytkownika B (tzn. B zablokował możliwość interakcji z użytkownikiem A)
        \item Nastąpił błąd komunikacji z serwerem --- nie udało się wysłać wiadomości.
        \item Serwer zasygnalizował błąd podczas dopisywania wiadomości do historii konwersacji.
    \end{enumerate}


    \subsection{Możliwe rozwinięcia}
    \subsubsection{Przesyłanie obrazów}
    Umożliwienie przesyłania wiadomości w formie zdjęć. W tym celu użytkownik wysyłający wiadomość wybiera piktogram ``prześlij obraz''. Dalszy scenariusz jest analogiczny do opisanego w punkcie 5.4.2.

    \section{Ocena innego użytkownika}

    \subsection{Krótki opis}
    Wystawienie innemu użytkownikowi oceny w skali od 1 do 5 oraz możliwość napisania krótkiego komentarza dotyczącego współpracy z nim.


    \subsection{Cele}
    Informowanie innych osób o jakości współpracy z daną osobą, w szczególności ostrzeganie innych przed nierzetelnymi użytkownikami.


    \subsection{Warunki wstępne}
    \begin{itemize}
        \item Aplikacja jest zainstalowana na urządzeniu użytkownika.
        \item Użytkownik posiada poprawnie założone oraz potwierdzone konto.
        \item Użytkownik jest zalogowany w aplikacji.
        \item Użytkownik dysponuje stabilnym połączeniem internetowym.
    \end{itemize}

    \subsection{Czynności}


    \subsubsection{Czynności podstawowe}


    \begin{enumerate}
        \item Użytkownik A otwiera ekran, na którym znajduje się publiczny profil użytkownika B. Może to zrobić na kilka sposobów:
        \begin{itemize}
            \item W oknie konwersacji z użytkownikiem B, poprzez kliknięcie na jego imię lub zdjęcie w lewym górnym rogu ekranu.
            \item Naciskając przycisk ``Pokaż profil użytkownika’’ na ekranie ogłoszenia, które użytkownik B wystawił.
            \item Wpisując: imię, nazwisko lub nazwę użytkownika B w polu tekstowym znajdującym się na samej górze ekranu prostej wyszukiwarki.
        \end{itemize}
        \item Wyświetla się ekran z profilem użytkownika B, a w nim następujące elementy:
        \begin{itemize}
            \item Imię, nazwisko oraz nazwa użytkownika.
            \item Zdjęcie profilowe.
            \item Zbiorcza ocena innych użytkowników w skali od 1 do 5.
            \item Przycisk ``Rozpocznij czat’’.
            \item Przycisk ``Dodaj ocenę’’.
            \item Lista z ocenami i komentarzami innych osób.
        \end{itemize}
        \item Użytkownik A naciska przycisk ``Dodaj ocenę’’.
        \item Wyświetla mu się ekran z formularzem dodawania oceny.
        \item Użytkownik A ocenia użytkownika B w skali od 1 do 5.
        \item Użytkownik A wpisuje krótki komentarz odnośnie użytkownika B.
        \item Użytkownik A naciska przycisk ``Dodaj ocenę’’.
    \end{enumerate}



    \subsubsection{Czynności alternatywne}
    Użytkownika A może co najwyżej raz ocenić użytkownika B. Jeśli miało już to miejsce, użytkownik A może edytować swoją opinię:
    \begin{enumerate}
        \item Zamiast przycisku ``Dodaj ocenę’’ na ekranie z profilem użytkownika B będzie widoczny przycisk ``Edytuj ocenę’’.
        \item Formularz dodawania oceny będzie uzupełniony o poprzednią ocenę i komentarz.
        \item Użytkownik po edycji formularza klika przycisk ``Zapisz ocenę’’.
    \end{enumerate}


    \subsubsection{Czynności niepoprawne}
    We wszystkich wymienionych poniżej przypadkach na ekranie wyświetla się (jaskrawym czerwonym kolorem) stosowny komunikat. Użytkownik zostaje
    przekierowany z powrotem do ekranu edycji formularza, aby mógł wprowadzić zmiany.
    \begin{enumerate}
        \item Nastąpił błąd komunikacji z serwerem --- prośba o dodanie / edycję opinii nie została rozpatrzona.
        \item Serwer zasygnalizował błąd podczas operacji dodawania / edycji opinii do bazy danych.
    \end{enumerate}


    \subsection{Możliwe rozwinięcia}
    Blokada możliwości oceny danego użytkownika do momentu, w którym odbierzemy od niego (lub on od nas) pożywienie z przynajmniej jednego ogłoszenia.


    \section{Wyświetlenie oraz edycja własnego profilu publicznego}

    \subsection{Krótki opis}
    Wyświetlenie ekranu ze swoim profilem publicznym, tak jak widzą go inni użytkownicy.


    \subsection{Cele}
    Umożliwienie użytkownikowi zapoznanie się z ocenami wystawionymi przez inne osoby oraz możliwość zmiany swojego zdjęcia profilowego.


    \subsection{Warunki wstępne}
    \begin{itemize}
        \item Aplikacja jest zainstalowana na urządzeniu użytkownika.
        \item Użytkownik posiada poprawnie założone oraz potwierdzone konto.
        \item Użytkownik jest zalogowany w aplikacji.
        \item Użytkownik dysponuje stabilnym połączeniem internetowym.
    \end{itemize}

    \subsection{Czynności}

    \subsubsection{Czynności podstawowe}

    \begin{enumerate}
        \item Użytkownik naciska oznaczony przycisk, który rozwija szufladkowe menu.
        \item Użytkownik wskazuje w menu pozycję ``Mój profil''.
        \item Na ekranie wyświetlają się następujące elementy:
        \begin{itemize}
            \item Imię, nazwisko oraz nazwa użytkownika.
            \item Zdjęcie profilowe.
            \item Przycisk ``Zmień zdjęcie’’
            \item Zbiorcza ocena innych użytkowników w skali od 1 do 5.
            \item Lista z ocenami i komentarzami innych osób.
        \end{itemize}
        \item Użytkownik A przewija ekran w dół, aby zapoznać się ze wszystkimi opiniami.
    \end{enumerate}

    \subsubsection{Czynności alternatywne}
    Użytkownika może zmienić swoje zdjęcie profilowe:
    \begin{enumerate}
        \item Użytkownik naciska przycisk ``Zmień zdjęcie’’
        \item Aplikacja wyświetla okienko z prośbą o wskazanie źródła zdjęcia, do wyboru: załączenie zdjęcia z pamięci urządzenia (galerii)
        lub utworzenie aparatem nowego zdjęcia.
        \begin{enumerate}
            \item Jeżeli użytkownik wyrazi chęć dodania istniejącego zdjęcia, aplikacja uruchamia klasyczny interfejs przeglądania galerii
            systemu operacyjnego.
            \item Wybór opcji zrobienia nowego zdjęcia przekazuje sterowanie wbudowanej w system operacyjny aplikacji obsługującej aparat.
        \end{enumerate}
        \item Jeżeli zdjęcie ma zbyt duży rozmiar, następuje jego kompresja (po stronie aplikacji).
        \item Na ekranie profilu pojawi się nowo zmienione zdjęcie.
        \item Użytkownik naciska przycisk ``Zapisz zmiany’’ w prawym górnym rogu ekranu.
        \item Zdjęcie zostaje wysłane na serwer.
    \end{enumerate}


    \subsubsection{Czynności niepoprawne}
    We wszystkich wymienionych poniżej przypadkach na ekranie wyświetla się (jaskrawym czerwonym kolorem) stosowny komunikat. Użytkownik zostaje
    przekierowany z powrotem do ekranu edycji formularza, aby mógł wprowadzić zmiany.
    \begin{enumerate}
        \item Nastąpił błąd komunikacji z serwerem --- prośba o dodanie zdjęcia nie została rozpatrzona.
        \item Serwer zasygnalizował błąd podczas operacji dodawania zdjęcia.
    \end{enumerate}

    \subsection{Możliwe rozwinięcia}
    Możliwość dodania innych informacji do publicznego profilu, na przykład:
    \begin{itemize}
        \item Link do profilu na Facebooku, Instagramie itp.
        \item Numer telefonu.
        \item Adres e-mail.
        \item Krótkiego komentarza o sobie.
    \end{itemize}




        \section{Odbiór przedmiotów oferowanych w liście ogłoszeń}

    \subsection{Krótki opis}

    Umożliwienie użytkownikowi, który wystawił ogłoszenie (A) i użytkownikowi, który jest zainteresowany odbiorem przedmiotu z ogłoszenia (B) ustalenia szczegółów odbioru.


    \subsection{Cele}
    \begin{itemize}
        \item Umożliwienie użytkownikom ustalenia szczegółów odbioru przedmiotu (np. czas odbioru, miejsce odbioru).
        \item Umożliwienie użytkownikowi zarządzania statusem swoich ogłoszeń.
    \end{itemize}


    \subsection{Warunki wstępne}
    \begin{itemize}
        \item Aplikacja jest zainstalowana na urządzeniach użytkowników A i B.
        \item Użytkownicy posiadają poprawnie założone oraz potwierdzone konta.
        \item Użytkownicy są zalogowani w aplikacji.
        \item Użytkownik B dodał ogłoszenie.
        \item Użytkownicy dysponują stabilnym połączeniem internetowym.
    \end{itemize}

    \subsection{Czynności}

    \subsubsection{Czynności podstawowe}

    \paragraph{Użytkownik A}

    \begin{enumerate}
        \item Użytkownik naciska oznaczony przycisk ``Szczegóły ogłoszenia'' przy ogłoszeniu na liście ogłoszeń.
        \item Na ekranie ze szczegółami ogłoszenia użytkownik naciska przycisk ``Zadaj pytanie''.
        \item Użytkownikowi otwiera się ekran czatu z autorem ogłoszenia.
        \item Użytkownik wysyła wiadomość, w której wyraża chęć odbioru przedmiotu z ogłoszenia.
        \item Jeżeli użytkownik otrzyma odpowiedź pozytywną, użytkownicy ustalają między sobą szczegóły odbioru.
    \end{enumerate}

    \paragraph{Użytkownik B}

    \begin{enumerate}
        \item Użytkownik otrzymuje nową wiadomość na czacie z chęcią odbioru oferowanego przedmiotu.
        \item Użytkownik decyduje, czy chce oddać oferowany produkt użytkownikowi A
        \begin{enumerate}
            \item Jeżeli użytkownik nie chce oddać produktu użytkownikowi A, wysyła wiadomość odmowną i interakcja się kończy.
            \item Jeżeli użytkownik chce oddać produkt, wysyła wiadomość pozytywną, interakcja trwa dalej.
        \end{enumerate}
        \item Na ekranie szczegółów swojego ogłoszenia użytkownik naciska przycisk ``Oznacz jako zarezerwowane''
        \item Użytkownicy ustalają między sobą szczegóły odbioru.
        \item Po oddaniu oferowanego przedmiotu, użytkownik naciska przycisk ``Oznacz jako oddane'' na ekranie szczegółów swojego ogłoszenia, przez co ogłoszenie znika z listy ogłoszeń.
    \end{enumerate}

    \subsubsection{Czynności alternatywne}
    Jeżeli po oznaczeniu ogłoszenia jako zarezerwowane, użytkownik A albo B zmieni zdanie co do chęci odbioru/oddania produktu, użytkownik B może odznaczyć ogłoszenie jako zarezerwowane poprzez ponowne kliknięcie w przycisk ``oznacz jako zarezerwowane'' na ekranie szczegółów swojego ogłoszenia.


    \subsubsection{Czynności niepoprawne}
    We wszystkich wymienionych poniżej przypadkach na ekranie wyświetla się (jaskrawym czerwonym kolorem) stosowny komunikat.
    \begin{enumerate}
        \item Czynności niepoprawne z podpunktu 6.4.2 dotyczące funkcjonalności czatu
        \item Nastąpił błąd komunikacji z serwerem - prośba o oznaczenie jako zarezerwowane/oddane nie została rozpatrzona.
        \item Serwer zasygnalizował błąd podczas operacji związanych z oznaczeniem ogłoszenia jako zarezerwowane/oddane.
    \end{enumerate}

    \subsection{Możliwe rozwinięcia}
    Podczas oznaczania ogłoszenia jako zarezerwowane użytkownik może oznaczyć, dla którego użytkownika dokonywana jest rezerwacja.



\end{document}






