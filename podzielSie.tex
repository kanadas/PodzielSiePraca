%
% Niniejszy plik stanowi przykład formatowania pracy magisterskiej na
% Wydziale MIM UW.  Szkielet użytych poleceń można wykorzystywać do
% woli, np. formatujac wlasna prace.
%
% Zawartosc merytoryczna stanowi oryginalnosiagniecie
% naukowosciowe Marcina Wolinskiego.  Wszelkie prawa zastrzeżone.
%
% Copyright (c) 2001 by Marcin Woliński <M.Wolinski@gust.org.pl>
% Poprawki spowodowane zmianami przepisów - Marcin Szczuka, 1.10.2004
% Poprawki spowodowane zmianami przepisow i ujednolicenie
% - Seweryn Karłowicz, 05.05.2006
% Dodanie wielu autorów i tłumaczenia na angielski - Kuba Pochrybniak, 29.11.2016

% dodaj opcję [licencjacka] dla pracy licencjackiej
% dodaj opcję [en] dla wersji angielskiej (mogą być obie: [licencjacka,en])
\documentclass[licencjacka]{pracamgr}

%\autor{Autor Zerowy}{342007}
\autori{Paweł Giżka}{TODO}
\autorii{Szymon Gajda}{TODO}
\autoriii{Kamil Ćwintal}{TODO}
\autoriv{Tomasz Kanas}{385674}
%\autorv{Autor nr Pięć}{342011}

\title{Mobilna aplikacja do dzielenia się żywnością ``Podziel się''}


%\tytulang{An implementation of a difference blabalizer based on the theory of $\sigma$ -- $\rho$ phetors}

%kierunek:
% - matematyka, informacyka, ...
% - Mathematics, Computer Science, ...
\kierunek{informatyka}

% informatyka - nie okreslamy zakresu (opcja zakomentowana)
% matematyka - zakres moze pozostac nieokreslony,
% a jesli ma byc okreslony dla pracy mgr,
% to przyjmuje jedna z wartosci:
% {metod matematycznych w finansach}
% {metod matematycznych w ubezpieczeniach}
% {matematyki stosowanej}
% {nauczania matematyki}
% Dla pracy licencjackiej mamy natomiast
% mozliwosc wpisania takiej wartosci zakresu:
% {Jednoczesnych Studiow Ekonomiczno--Matematycznych}

% \zakres{Tu wpisac, jesli trzeba, jedna z opcji podanych wyzej}

% Praca wykonana pod kierunkiem:
% (podać tytuł/stopień imię i nazwisko opiekuna
% Instytut
% ew. Wydział ew. Uczelnia (jeżeli nie MIM UW))
\opiekun{dra Janusza Jabłonowskiego\\
  Instytut Informatyki\\
  }

% miesiąc i~rok:
\date{Czerwiec 2019}

%Podać dziedzinę wg klasyfikacji Socrates-Erasmus:
\dziedzina{
%11.0 Matematyka, Informatyka:\\
%11.1 Matematyka\\
%11.2 Statystyka\\
11.3 Informatyka\\
%11.4 Sztuczna inteligencja\\
%11.5 Nauki aktuarialne\\
%11.9 Inne nauki matematyczne i informatyczne
}

%Klasyfikacja tematyczna wedlug AMS (matematyka) lub ACM (informatyka)
\klasyfikacja{D. Software\\
}

% Słowa kluczowe:
\keywords{aplikacja mobilna, żywność, ReactNative}

% Tu jest dobre miejsce na Twoje własne makra i~środowiska:
\newtheorem{defi}{Definicja}[section]

% koniec definicji

\begin{document}

\maketitle

%tu idzie streszczenie na strone poczatkowa
\begin{abstract}
\textit{``Podziel się''} to aplikacja mobilna przeznaczona do wzajemnego dzielenia się żywnością dla użytkowników systemów operacyjnych Android oraz iOS.\ Celem projektu było stworzenie platformy, która umożliwi łatwe i bezpłatne dzielenie się jedzeniem z innymi mieszkańcami w okolicy: osoby posiadające nadwyżkę artykułów żywnościowych lub gotowych potraw mogą przekazać je użytkownikom, którzy zadeklarują chęć ich odbioru. W niniejszej pracy przedstawiono szczegółowy opis aplikacji i jej architektury, oraz przybliżono proces jej powstawania.
\end{abstract}

\tableofcontents
%\listoffigures
%\listoftables

\chapter*{Wprowadzenie}
\addcontentsline{toc}{chapter}{Wprowadzenie}
\section*{Problematyka}
Obecnie w Polsce istnieją liczne grupy entuzjastów wymiany żywności funkcjonujące na portalach społecznościowych (przeważnie na Facebooku), jednak duża liczebność tych grup uniemożliwia skuteczną organizację wymiany jedzenia w satysfakcjonujący sposób. Aplikacja \textit{``Podziel się''} będzie wspierać dalsze funkcjonowanie tej inicjatywy oraz zapewni zwolennikom idei food sharingu platformę do efektywnej wymiany żywności, także w znacznie większej skali.

\section*{Kontekst społeczny}
Według współczesnych szacunków, około 1/3 wytworzonej na świecie żywności nigdy nie zostanie zjedzona. Skala problemu jest szczególnie widoczna w danych Eurostatu, z których wynika, że w Polsce marnuje się około 9 milionów ton żywności rocznie. Jej wartość sięga około 50 złotych miesięcznie w przeliczeniu na~jednego mieszkańca Polski. Badanie Millward Brown wykazało, że do wyrzucania jedzenia przyznaje się 35\% Polaków, podczas gdy wyrzucane produkty mogłyby przyczynić się do zaspokojenia podstawowych potrzeb żywieniowych niemal 3 mln osób żyjących w skrajnym ubóstwie. Nasz kraj jest piątym wśród państw Unii Europejskiej, w których marnuje się najwięcej żywności.

W krajach rozwiniętych około 50\% wyrzucanego jedzenia pochodzi z gospodarstw domowych, w których na skutek nierozważnego planowania zakupów jedzenie ulega zepsuciu i~trafia do śmietnika. Wierzymy, że~możliwe jest przeciwdziałanie temu zjawisku na poziomie lokalnym.

Istnieją ruchy społeczne (dumpster diving, freeganizm, zero waste) promujące możliwie racjonalny obrót żywnością oraz uwrażliwiające społeczeństwo na problem masowego wyrzucania jedzenia.

\section*{Zamawiający}
Aplikacja \textit{``Podziel się''} powstała na zamówienie pani Marii Skołożyńskiej, działaczki społecznej oraz pomysłodawczyni akcji \textit{Podziel się Posiłkiem z Bezdomnymi}, podczas której mieszkańcy całej Polski mogą przekazać pozostałe po świętach (Bożego Narodzenia, Wielkanocy) posiłki do wolontariuszy, którzy zajmują się transportem jedzenia do osób potrzebujących. Pani Maria jest wolontariuszką działającą na rzecz przeciwdziałania marnowaniu jedzenia oraz główną koordynatorką akcji dystrybucji żywności. Działalność pani Marii Skołożyńskiej ma zasięg ogólnopolski; akcje gromadzenia żywności dla bezdomnych i ubogich działają obecnie w kilkudziesięciu miastach, m.\ in.\ w Warszawie, Krakowie, Wrocławiu, Trójmieście, Szczecinie, Toruniu, Bydgoszczy, Katowicach, Łodzi. W 2018 r.\ akcja \textit{Podziel się Posiłkiem z Bezdomnymi}, w~uznaniu dla skuteczności działania oraz promocji pozytywnych postaw społecznych, otrzymała Europejską Nagrodę Reduce Food Waste.

\section*{Osoby zaangażowane w projekt}
Aplikacja \textit{``Podziel się''} została opracowana oraz wdrożona przez zespół studentów informatyki Uniwersytetu Warszawskiego: Kamila Ćwintala, Szymona Gajdę, Pawła Giżkę oraz Tomasza Kanasa. Opiekę nad projektem sprawował dr Janusz Jabłonowski. Wymagania funkcjonalne zostały dostarczone zespołowi przez pomysłodawczynie projektu: p. Marię Skołożyńską oraz p. Martynę Dakowską.

\section*{Grupa docelowa}
Aplikacja \textit{``Podziel się''} jest przeznaczona dla użytkowników smartfonów, którym nie jest obojętny problem marnowania żywności. \textit{``Podziel się''} może stanowić interesującą propozycję dla osób, które nie mają czasu na przygotowanie posiłku albo gotują zbyt dużo (i~jednocześnie nie chcą wyrzucać nadwyżki jedzenia). Aplikacja \textit{``Podziel się''} zrzesza społeczność osób zainteresowanych tematyką food sharingu i zero waste, które dbają o środowisko naturalne oraz racjonalną gospodarkę żywnością.

\section*{Opis funkcjonalności}
Aplikacja \textit{``Podziel się''} umożliwia:
\begin{itemize}
\setlength\itemsep{-0.2em}
\item bezpłatną rejestrację nowego konta,
\item uzupełnienie własnego profilu o podstawowe dane personalne,
\item dodawanie nowego ogłoszenia (opisu produktu wraz z preferowaną lokalizacją miejsca odbioru),
\item zamieszczenie w ogłoszeniu zdjęć produktu, opakowania, etykiet itp.,
\item przeglądanie dostępnych do odebrania posiłków w okolicy aktualnej lokalizacji użytkownika,
\item filtrowanie dostępnych ogłoszeń w oparciu o wprowadzone przez użytkownika kryteria i preferencje,
\item funkcję czatu między odbiorcą a darczyńcą w celu ustalenia szczegółów odbioru, zadawania pytań,
\item przechowywanie historii konwersacji z innymi użytkownikami,
\item wskazanie użytkownika (ze zbiorczej listy osób zainteresowanych ogłoszeniem), która otrzyma wystawiony~produkt/artykuł żywnościowy,
\item pośrednictwo w procesie przekazania żywności,
%TODO czy to jest zrobione?
\item usuwanie ogłoszeń zrealizowanych pomyślnie lub ogłoszeń, dla których minęła data przydatności do~spożycia,
\item system recenzji/ocen użytkowników,
%TODO tego nie ma nie?
\item wyróżnianie wybranych użytkowników jako ``ulubionych'', dzięki czemu ich nowe ogłoszenia będą dodatkowo eksponowane na liście dostępnych ogłoszeń,
%TODO czy to jest zrobione?
\item flagowanie ogłoszeń o nieodpowiedniej zawartości, co skutkuje powiadomieniem moderatora o~konieczności~weryfikacji zamieszczonej treści,
%TODO tego nie ma nie?
\item personalizację wyglądu aplikacji (np.\ kolor elementów interfejsu).
\end{itemize}
Aplikację wyróżnia nowoczesny interfejs graficzny, zaprojektowany specjalnie na potrzeby projektu.

%cytowanie: \cite
\subsection*{Zawartość pracy}
W rozdziale~\ref{r:konkurencja} przedstawiono przegląd i analizę konkurencyjnych rozwiązań. Rozdział~\ref{r:arch} zawiera opis użytych technologi i architektury systemu. W kolejnych rozdziałach opisano implementację aplikacji i wybrane przypadki użycia. Rozdział~\ref{r:problem} opisuje problemy i trudności które napotkaliśmy podczas tworzenia aplikacji oraz jak sobie z nim poradziliśmy. Ostatnie rozdziały zawierają instrukcje budowania aplikacji i zawartość płytki dołączonej do pracy, podział pracy w zespole i krótkie podsumowanie pracy.

\chapter{Przegląd konkurencyjnych rozwiązań}\label{r:konkurencja}
\chapter{Użyte technologie i architektura systemu}\label{r:arch}
\chapter{Implementacja}\label{r:impl}
\chapter{Przypadki użycia}\label{r:usecase}
\chapter{Problemy i trudności podczas tworzenia aplikacji}\label{r:problem}
\chapter{Instrukcja budowania aplikacji i zawartość płytki}\label{r:build}
\chapter{Podział pracy}\label{r:podzial}
\chapter{Podsumowanie}\label{r:pods}
\begin{thebibliography}{99}
\addcontentsline{toc}{chapter}{Bibliografia}

\bibitem[Bea65]{beaman} Juliusz Beaman, \textit{Morbidity of the Jolly
    function}, Mathematica Absurdica, 117 (1965) 338--9.

\bibitem[Blar16]{eb1} Elizjusz Blarbarucki, \textit{O pewnych
    aspektach pewnych aspektów}, Astrolog Polski, Zeszyt 16, Warszawa
  1916.

\bibitem[Fif00]{ffgg} Filigran Fifak, Gizbert Gryzogrzechotalski,
  \textit{O blabalii fetorycznej}, Materiały Konferencji Euroblabal
  2000.

\bibitem[Fif01]{ff-sr} Filigran Fifak, \textit{O fetorach
    $\sigma$-$\rho$}, Acta Fetorica, 2001.

\bibitem[Głomb04]{grglo} Gryzybór Głombaski, \textit{Parazytonikacja
    blabiczna fetorów --- nowa teoria wszystkiego}, Warszawa 1904.

\bibitem[Hopp96]{hopp} Claude Hopper, \textit{On some $\Pi$-hedral
    surfaces in quasi-quasi space}, Omnius University Press, 1996.

\bibitem[Leuk00]{leuk} Lechoslav Leukocyt, \textit{Oval mappings ab ovo},
  Materiały Białostockiej Konferencji Hodowców Drobiu, 2000.

\bibitem[Rozk93]{JR} Josip A.~Rozkosza, \textit{O pewnych własnościach
    pewnych funkcji}, Północnopomorski Dziennik Matematyczny 63491
  (1993).

\bibitem[Spy59]{spyrpt} Mrowclaw Spyrpt, \textit{A matrix is a matrix
    is a matrix}, Mat. Zburp., 91 (1959) 28--35.

\bibitem[Sri64]{srinis} Rajagopalachari Sriniswamiramanathan,
  \textit{Some expansions on the Flausgloten Theorem on locally
    congested lutches}, J. Math.  Soc., North Bombay, 13 (1964) 72--6.

\bibitem[Whi25]{russell} Alfred N. Whitehead, Bertrand Russell,
  \textit{Principia Mathematica}, Cambridge University Press, 1925.

\bibitem[Zen69]{heu} Zenon Zenon, \textit{Użyteczne heurystyki
    w~blabalizie}, Młody Technik, nr~11, 1969.

\end{thebibliography}

\end{document}


%%% Local Variables:
%%% mode: latex
%%% TeX-master: t
%%% coding: latin-2
%%% End:
